\chapter{Ano Letivo 2023/2024}
\label{cap3}

\section{Expansão da Base de Dados}
Foi realizado um processo de ETL (Extração, Transformação e Carregamento) de dados de produtos portugueses a partir do site OpenFoodFacts. Scripts em \texttt{Python} e \texttt{Node.js} foram utilizados para filtrar, transformar e importar os dados para o Firebase. Foi introduzido o atributo \texttt{isActive} para gestão dos códigos de barras.

\section{Melhorias na Interface do Utilizador}
\begin{itemize}
    \item Remoção de botões redundantes como ``Voltar Atrás''.
    \item Renomeação do botão ``A Que Equivalem'' para ``Referências'', tornando mais clara a sua função.
    \item Correções gramaticais e simplificação da navegação.
\end{itemize}

\section{Interface do Administrador}
\begin{itemize}
    \item Redesenho da página de estatísticas com filtros mais flexíveis.
    \item Criação de menus e botões adicionais para validação e adição de códigos de barras.
    \item Visualização de dados gerais sem obrigatoriedade de seleção de filtros.
\end{itemize}

\section{Sistema de Recompensas}
Foi introduzido o conceito de recompensas para incentivar a participação dos utilizadores. Foram desenvolvidas interfaces para:
\begin{itemize}
    \item Visualização de recompensas.
    \item Detalhes de cada recompensa.
    \item Partilha de conquistas nas redes sociais.
\end{itemize}

\section{Reconhecimento de Objetos}
Foi identificada uma baixa taxa de sucesso no reconhecimento por imagem estática. A equipa sugeriu a substituição por reconhecimento via vídeo em futuras versões.

\section{Testes Internos}
Foram realizados testes internos que revelaram discrepâncias em relação aos resultados anteriores, especialmente na taxa de sucesso do reconhecimento de objetos. As melhorias foram aplicadas com base nesses testes.

\section{Limitações Persistentes}
\begin{itemize}
    \item Estatísticas ambientais ainda pouco detalhadas (sem considerar volume).
    \item Redundâncias na base de dados.
    \item Sistema de recompensas ainda em fase inicial, com poucas opções disponíveis.
    \item Dependência de conectividade com a internet para acesso a funcionalidades completas.
\end{itemize}

\section{Conclusão}
O ano letivo de 2023/2024 trouxe desafios significativos, mas também oportunidades de crescimento e aprendizagem. Agradecemos a todos os que contribuíram para o desenvolvimento e melhoria contínua deste projeto.


