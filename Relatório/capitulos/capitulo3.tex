\chapter{Ano Letivo 2023/2024}
\label{cap3}

\section{Expansão da Base de Dados}
Executou-se um processo de ETL (Extração, Transformação e Carregamento) com dados de produtos portugueses do OpenFoodFacts. Usaram-se scripts em \texttt{Python} e \texttt{Node.js} para filtrar, transformar e carregar os dados no Firebase. Introduziu-se o atributo \texttt{isActive} para gerir os códigos de barras.

\section{Melhorias na Interface do Utilizador}
\begin{itemize}
    \item Remoção de botões redundantes, como ``Voltar Atrás''.
    \item O botão ``A Que Equivalem'' passou a ``Referências'', tornando a função mais clara.
    \item Correções linguísticas e simplificação da navegação.
\end{itemize}

\section{Interface do Administrador}
\begin{itemize}
    \item Redesenho da página de estatísticas, com filtros mais flexíveis.
    \item Criação de menus e botões para validar e adicionar códigos de barras.
    \item Possibilidade de ver dados gerais sem necessidade de selecionar filtros.
\end{itemize}

\section{Sistema de Recompensas}
Introduziu-se o conceito de recompensas para incentivar a participação dos utilizadores. Foram criadas interfaces para:
\begin{itemize}
    \item Visualização de recompensas.
    \item Detalhes de cada recompensa.
    \item Partilha de conquistas nas redes sociais.
\end{itemize}

\section{Reconhecimento de Objetos}
Verificou-se uma taxa de sucesso baixa no reconhecimento por imagem estática. A equipa sugeriu a substituição por reconhecimento em vídeo em versões futuras.

\section{Testes Internos}
Realizaram-se testes internos que evidenciaram diferenças face a resultados anteriores, sobretudo na taxa de sucesso do reconhecimento de objetos. As melhorias foram aplicadas com base nesses testes.

\section{Limitações Persistentes}
\begin{itemize}
    \item Estatísticas ambientais ainda pouco detalhadas (sem considerar volume).
    \item Redundâncias na base de dados.
    \item Sistema de recompensas ainda em fase inicial, com poucas opções.
    \item Dependência de ligação à internet para acesso a todas as funcionalidades.
\end{itemize}

\section{Conclusão}
O ano letivo de 2023/2024 trouxe desafios e oportunidades de melhoria. Agradece-se a todos os que contribuíram para o desenvolvimento e a evolução contínua deste projeto.


