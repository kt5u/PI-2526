\chapter{Questionário de Avaliação da Aplicação ReciclaApp}
\label{cap8}

\section{Objetivo}
O principal objetivo deste questionário é perceber como é que professores, docentes e funcionários 
do IPBeja realmente utilizam a aplicação e o que pensam sobre ela. Interessa-nos saber se a ReciclaApp
lhes faz sentido no dia a dia, se é fácil de usar, se ajuda de alguma forma no processo de reciclagem 
e se existe vontade de continuar a utilizá-la no futuro. 

Isto permite reunir informação de forma mais consistente e comparar melhor a experiência dos diferentes utilizadores.
As respostas recolhidas são muito importantes para entendermos o que está bem encaminhado e o que ainda precisa 
de ser melhorado. Estas opiniões ajudam não só a confirmar se as funcionalidades atuais estão a cumprir o que se 
pretende, mas também a orientar o trabalho futuro. Assim, a aplicação pode ir evoluindo com base em feedback real de quem a utiliza.
Para realização do questionário foi usado o GoogleForms.
\clearpage
\section{Escala de resposta}
A maior parte das questões usa uma escala Likert de 5 pontos:

\begin{description}
  \item[1] Discordo totalmente
  \item[2] Discordo
  \item[3] Nem concordo nem discordo
  \item[4] Concordo
  \item[5] Concordo totalmente
\end{description}

\section{Modelo}
O questionário foi elaborado com base no modelo UTAUT2 (Unified Theory of Acceptance and Use of Technology 2), 
um dos referenciais teóricos mais utilizados para compreender os fatores que influenciam a aceitação e o uso de tecnologias. 
 
Dessa forma, o questionário procura captar de forma abrangente os elementos que determinam a intenção de uso e a adoção 
efetiva da tecnologia pelos indivíduos, oferecendo uma base sólida para a análise dos resultados.

O questionário pode ser acedido através do seguinte link: \url{https://forms.gle/2bKXGUvXgRwZpfGE9}

De seguida, é apresentado o questionário, que poderá ser imprimido e que poderá também ser realizado em papel. 
Este documento encontra-se em anexo ao relatório.

\includepdf[pages=-]{figuras/Questionario_ReciclaApp.pdf}
