\chapter{Caracterização do Ambiente, Objetivos de Amostragem e Locais de Teste}
\label{cap7}

Para o planeamento rigoroso dos testes de campo da aplicação, é fundamental compreender o ambiente onde estes serão realizados, contemplando fatores que influenciam o desempenho e a eficácia da aplicação, como a infraestrutura tecnológica, o perfil dos utilizadores, as condições ambientais e os objetivos de amostragem.

\section{Caracterização do ambiente
digital}

Considerando que a aplicação foi desenvolvida exclusivamente para dispositivos Android, torna-se essencial garantir que os testes considerem a fragmentação do ecossistema Android, que inclui variações significativas nas versões do sistema operativo e tamanhos de ecrã. A definição clara das versões mínimas e alvo suportadas pela aplicação é indispensável para a avaliação do seu desempenho e usabilidade nos dispositivos efetivamente presentes no mercado académico e campus do IPBeja.

Como etapa inicial, serão realizados testes em emuladores do Android Studio, simulando diferentes níveis de API e tamanhos de ecrã para validar a compatibilidade básica, identificar problemas de layout, desempenho ou falhas. Posteriormente, os testes deverão continuar em dispositivos físicos Android disponíveis no campus (de alunos, docentes e funcionários), assegurando cobertura dos perfis de hardware e software mais representativos do ecossistema Android presente no IPBeja.

\section{Número e perfil dos utilizadores}

Para uma avaliação representativa e abrangente, é importante definir o número e o perfil dos utilizadores que participarão nos testes. A amostra deve refletir a diversidade da comunidade académica do IPBeja, incluindo estudantes de diferentes cursos e anos, docentes com variadas áreas e funcionários. Esta diversidade permitirá recolher feedback rico, refletindo necessidades e experiências distintas.

Idealmente, a amostra deve ter entre 30 a 50 utilizadores, equilibradamente distribuídos entre estudantes, docentes e funcionários. Este tamanho permite obter dados estatisticamente significativos para identificar padrões de utilização, preferências e áreas para melhoria. Deve ainda incluir utilizadores com variados níveis de familiaridade tecnológica, garantindo a avaliação da usabilidade e acessibilidade em múltiplos contextos.

\section{Objetivos de amostragem}

Os principais objetivos de amostragem para os testes de campo visam:

\begin{itemize}
\item Avaliar a usabilidade da aplicação em diferentes dispositivos Android, assegurando uma interface intuitiva para todos os utilizadores.
\item Medir o desempenho em termos de velocidade, estabilidade e funcionamento sob diversas condições de conectividade.
\item Recolher feedback qualitativo e quantitativo sobre a experiência geral, funcionalidades e sugestões.
\item Analisar o impacto na promoção da reciclagem no campus, verificando motivação e sensibilização dos utilizadores.
\item Identificar bugs e problemas técnicos para correção.
\item Compreender as preferências relativas a funcionalidades como notificações, gamificação e integração social.
\item Avaliar a aceitação da aplicação, identificando barreiras e estratégias para aumentar o envolvimento.
\end{itemize}

\section{Local de realização dos testes}

Os testes de campo da aplicação serão realizados não apenas no campus do Instituto Politécnico de Beja (IPBeja), mas também em ambientes domésticos, a fim de maximizar a adesão dos utilizadores ao estudo. Permitir que os participantes possam utilizar a aplicação em casa, além dos espaços comuns do campus como bibliotecas, refeitórios ou zonas de lazer, favorece a inclusão de um maior número e diversidade de utilizadores que possam integrá-la mais facilmente em suas rotinas diárias.

Esta abordagem flexível possibilita a avaliação da aplicação em contextos reais e variados, refletindo com maior fidelidade as situações cotidianas de uso e permitindo testar o desempenho da aplicação sob diferentes condições de rede e ambiente. O teste em ambiente doméstico também oferece insights valiosos sobre o comportamento prolongado dos utilizadores e a aceitação da aplicação em situações menos controladas.

Além disso, a possibilidade de realizar testes fora do campus aumenta significativamente a probabilidade de maior engajamento e participação no estudo, reduzindo barreiras logísticas associadas à necessidade de deslocamento e aos horários restritos. Essa expansão do local de teste para ambientes domésticos contribui para obter uma amostra mais representativa e abrangente da comunidade académica, enriquecendo a qualidade do feedback e a robustez da avaliação da aplicação.