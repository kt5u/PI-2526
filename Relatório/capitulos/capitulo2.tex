\chapter{Ano Letivo 2022/2023}
\label{cap2}

\section{Continuidade e Validação do Projeto}

O projeto foi recebido de grupos anteriores e o foco principal foi validar o que já existia. A equipa começou por analisar a aplicação, fez testes com utilizadores e identificou problemas de usabilidade, desempenho e clareza. Definiram‑se melhorias simples e progressivas, dando prioridade à fiabilidade do registo de reciclagens e à redução da complexidade dos fluxos.

\section{Migração Tecnológica}

O código foi migrado de Java para Kotlin para melhorar a legibilidade, facilitar a manutenção e aumentar o desempenho. A mudança permitiu uma integração mais fácil com o Android Studio e com o Firebase. Em paralelo, reduziu‑se a dívida técnica: reorganização de pacotes, remoção de redundâncias e adoção de convenções idiomáticas de Kotlin.

\newpage
\section{Reestruturação da Base de Dados}

A base de dados Firebase foi reorganizada em coleções específicas:
\begin{itemize}
    \item \texttt{codigo\_barras}
    \item \texttt{reciclagens}
    \item \texttt{users}
\end{itemize}
Foram corrigidos problemas de estrutura e melhorada a ligação entre as reciclagens e os utilizadores.

\section{Reconhecimento Automático de Objetos}

Foram usados dois métodos complementares:
\begin{itemize}
    \item Leitura de códigos de barras com processamento de frames em tempo real, usando Google ML Kit e \texttt{CameraView}, para maior taxa de deteção e robustez.
    \item Classificação de imagem estática com um modelo de \emph{deep learning}.
\end{itemize}
Quando a câmara não detetava o código de barras, o utilizador podia sugerir o código para aprovação do administrador, evitando erros na inserção de dados.

\section{Funcionalidades Implementadas}

\begin{itemize}
    \item Autenticação de utilizadores e administradores.
    \item Registo de reciclagens por três métodos: código de barras, reconhecimento de objeto e inserção manual.
    \item Cálculo automático do impacto ambiental (CO\textsubscript{2}, energia e petróleo).
    \item Contextualização dos impactos (por exemplo, emissões evitadas e poupanças estimadas) para promover literacia ambiental.
    \item Visualização e validação de reciclagens pelo administrador.
\end{itemize}

\section{Interface Gráfica}

Foram criados novos layouts para login, registo e páginas de reciclagem. A área de administração passou a ter botões para validar e adicionar códigos de barras. Melhorou‑se a responsividade para ecrãs pequenos e de baixa resolução e simplificou‑se a interface de registo para reduzir erros.

\section{Testes e Validação}

A aplicação foi testada com 26 utilizadores (alunos e funcionários). Os dados recolhidos, quantitativos e qualitativos, suportaram melhorias na navegação, no design e no desempenho.

\section{Limitações Identificadas}

\begin{itemize}
    \item Erros de usabilidade e de tradução.
    \item Reconhecimento de objetos com fiabilidade reduzida.
    \item Redundância de imagens na base de dados.
    \item Compatibilidade limitada com versões antigas do Android.
\end{itemize}

\section{Conclusões e Trabalhos Futuros}

A validação foi globalmente positiva. Recomenda‑se continuar a otimizar a aplicação e a acrescentar funcionalidades com base no feedback dos utilizadores. Em particular, propõe‑se: (i) uma interface no módulo de administração para gerir códigos de barras pendentes de aprovação; (ii) melhorar o reconhecimento de objetos a partir de fotografia, para maior precisão

