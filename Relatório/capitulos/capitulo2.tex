\chapter{Ano Letivo 2022/2023}
\label{cap2}

\section{Análise inicial do Projeto Anterior(21/22)}

Foi realizado pelo grupo de alunos a avaliação da continuidade do projeto, começando pela análise detalhada do relatório e da aplicação do ano (21/22). Com a consequente validação, passaram para a fase de migração, aprimoramento e implementação de novas funcionalidades e melhorias de interface.

\section{Migração Tecnológica}

O código apresentado no relatório anterior constatava que o projeto (21/22) havia sido realizado na linguagem \texttt{Java}, e os alunos decidiram migrar para a linguagem \texttt{Kotlin} devido às suas vantagens, como a simplificação do código e a obtenção de uma estrutura mais simples e eficiente. A migração foi realizada com o auxílio do Android Studio, que possui uma ferramenta integrada para converter código Java em Kotlin. Após essa tentativa inicial de conversão, foi necessário realizar ajustes manuais para garantir que o código convertido funcionasse corretamente, incluindo a criação de fragmentos para cada interface e a diferenciação entre as classes de interface para utilizadores comuns e administradores.

\newpage
\section{Alterações na Base de Dados}
O grupo teve que realizar uma nova ligação á base de dados Firebase (Goob) e foi decidido manter a base de dados do projeto anterior devido a facilidade de integrações entre a firebase e Android, bem como a quantidade de dados já existentes relativos ao código de barras já existentes. A estrutura da base de dados foi mantida mas não foram utilizadas algumas coleções e foram adicionadas novas coleções para suportar as novas funcionalidades da aplicação.

\section{Alterações Implementadas}
Foram implementadas algumas alteraçoes e melhorias na aplicação, nomeadamente:
\subsection{Reconhecimento Automático de Objetos}
O grupo verificou que o reconhecimento de objetos apresentava alguns problemas, pois a aplicação não conseguia identificar corretamente certos objetos. Isso acontecia porque o sistema processava o código de barras apenas após a fotografia ser tirada, o que, por vezes, levava à necessidade de tirar várias fotos. Para resolver esse problema, foi implementado o sistema de ML Kit da Google, que permite o reconhecimento automático de objetos em tempo real através da câmara do telemóvel, utilizando a API CameraView. Para implementar essa API, foi utilizado um sistema de processamento de frames, possibilitando o uso do scanner em todos os frames capturados pela câmara.
\newpage

\subsection{Adição de Códigos de Barras}
No início do processo, o grupo optou por manter a funcionalidade de adicionar códigos de barras pelos utilizadores, mas verificou-se que o sistema ficaria suscetível a erros que estes poderiam cometer, como a inserção de dados incorretos relativos ao objeto. Com a sugestão dos docentes orientadores, o grupo decidiu implementar um sistema de validação por parte do administrador, onde os utilizadores poderiam adicionar os dados referentes ao código de barras e o administrador teria a possibilidade de validar ou rejeitar o código. Para isso, foi criada uma flag na base de dados chamada "isActive", que por defeito é falsa, passando a verdadeira quando o administrador valida o código de barras.

\clearpage

\begin{figure}
    \centering
    \includegraphics[width=0.2\textwidth]{figuras/NaoRecon.png}
    \caption{Não Reconhecimento do Código de Barras}
    \label{fig:codigosDeBarras}
\end{figure}

\begin{figure}
    \centering
    \includegraphics[width=0.2\textwidth]{figuras/SugestaoBarras.png}
    \caption{Sugestão de Código de Barras}
    \label{fig:codigosDeBarras2}
\end{figure}
\newpage

\subsection{Responsividade do Menu e Interfaces}
O grupo tentou resolver alguns problemas de responsividade que existiam em alguns ecrãs mais pequenos e baixas resoluções, alguma parte das interfaces ficavam cortadas ou desajustadas.Foram usados alguns metodos dos layouts do Android como a vertical e horizontal chain para melhorar a responsividade da aplicação.
Foram realizadas novas melhorias, incluindo a adição da contextualização para o utilizador conseguir entender melhor a sua reciclagem. 
Também foi melhorada a interface que exibe a percentagem de reciclagem de cada escola, apresentando métricas como a quantidade de objetos reciclados, energia poupada e mais indicadores relevantes.
Foram tambem alterados os textos informativos sobre o tipos de reciclagem e alguns botões da página inicial para melhorar a experiência do utilizador. 

Exemplo de responsividade / interfaces melhoradas: 
\begin{figure}
    \centering
    \includegraphics[width=0.4\textwidth]{figuras/ResponMelhor.png}
    \caption{Responsividade Melhorada}
    \label{fig:responsividade}
\end{figure}

\begin{figure}
    \centering
    \includegraphics[width=0.4\textwidth]{figuras/PercentagemEscolas.png}
    \caption{Interface Percentagem de Reciclagem das Escolas}
    \label{fig:responsividade2}
\end{figure}



\section{Testes e Validação}

Foram realizados testes funcionais e de usabilidade para validar as altereações implementadas.

Documentos das tarefas a realizar:
\begin{figure}
    \centering
    \includegraphics[width=0.6\textwidth]{figuras/testes.png}
    \caption{Documentos de Testes}
    \label{fig:testes}
\end{figure}

O grupo concluiu que a aplicação está funcional e pronta para ser utilizada pelos utilizadores, tendo a aplicação sido bem recebida pelos utilizadores durante os testes de usabilidade.
   

\section{Trabalhos Futuros}
O grupo sugere que futuros trabalhos possam focar se na implementação da funcionalidade da parte do administrador, que permita a gestão dos código de barras sugeridos pelos utilizadores e podendo ainda ser melhorado o sistema de reconhecimento de objetos.

