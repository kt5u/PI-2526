\chapter{Ano Letivo 2022/2023}
\label{cap2}

\section{Continuidade e Validação do Projeto}

O projeto foi herdado de grupos anteriores e teve como foco a validação do trabalho previamente realizado. A equipa iniciou com uma análise crítica da aplicação existente, seguida de testes com utilizadores para identificar problemas de usabilidade, desempenho e clareza funcional.

\section{Migração Tecnológica}

Foi realizada a migração da linguagem de programação Java para Kotlin, com o objetivo de melhorar a legibilidade do código, facilitar a manutenção e otimizar o desempenho da aplicação. Esta transição permitiu uma integração mais eficiente com o Android Studio e com a base de dados Firebase.

\newpage
\section{Reestruturação da Base de Dados}

A base de dados Firebase foi reorganizada em coleções específicas:

\begin{itemize}
    \item \texttt{codigo\_barras}
    \item \texttt{reciclagens}
    \item \texttt{users}
\end{itemize}

Foram corrigidos problemas de estrutura e melhorada a ligação entre os dados das reciclagens e os utilizadores.

\section{Funcionalidades implementadas}

\begin{itemize}
    \item Autenticação de utilizadores e administradores.
    \item Registo de reciclagens por três métodos: código de barras, reconhecimento de objeto e inserção manual.
    \item Cálculo automático do impacto ambiental (CO\textsubscript{2}, energia e petróleo).
    \item Visualização e validação de reciclagens por parte do administrador.
\end{itemize}

\section{Interface Gráfica}
Foram criados novos layouts para login, registo e páginas de reciclagem. A interface do administrador passou a incluir botões para validação e adição de códigos de barras.

\section{Testes e Validação}
A aplicação foi testada com 26 utilizadores (alunos e funcionários), tendo sido recolhidos dados quantitativos e qualitativos que permitiram melhorias na navegação, design e desempenho.

\section{Limitações Identificadas}
\begin{itemize}
    \item Erros de usabilidade e tradução.
    \item Reconhecimento de objetos com baixa fiabilidade.
    \item Redundância de imagens na base de dados.
    \item Dificuldades de integração com versões antigas do sistema operativo Android.
\end{itemize}

\section{Conclusões e Trabalhos Futuros}
A validação do projeto foi considerada um sucesso, com a maioria dos utilizadores a relatar uma experiência positiva. Para o futuro, recomenda-se a continuação da otimização da aplicação, bem como a adição de novas funcionalidades com base no feedback dos utilizadores.

