\chapter{Conclusões Finais}
\label{cap6}

\section{Síntese do Trabalho Realizado e Melhorias Implementadas}

\subsection{O que foi feito}
\begin{itemize}
    \item \textbf{Estudos de Campo:} Planeamento e execução de questionários pré-teste e pós-teste com utilizadores reais.
    \item \textbf{Análise de Resultados:} Construção de dashboards no \textit{Power BI} para visualização e interpretação dos dados.
    \item \textbf{Avaliação de Usabilidade:} Utilização de instrumentos como o \textit{SUS} e enquadramento teórico via \textbf{UTAUT2}.
    \item \textbf{Documentação:} Estruturação dos capítulos, contextualização metodológica e registo das principais conclusões por etapa.
\end{itemize}

\subsection{Principais melhorias na aplicação}
\begin{itemize}
    \item \textbf{Alinhamento conceptual:} Alteração de \textit{ReciclaApp} para \textit{EcoSepara} e atualização da terminologia de “reciclagem” para “separação de resíduos”.
    \item \textbf{Rastreabilidade:} Inclusão dos campos \texttt{user\_id} e \texttt{localizacao} (dentro/fora do IPBeja) na tabela de reciclagens para auditoria dos registos.
    \item \textbf{Interface de administração:} Exibição do utilizador autenticado nos pedidos de adição de códigos de barras.
\end{itemize}

\subsection{Principais resultados observados}
\begin{itemize}
    \item \textbf{Separação de resíduos:} Maioria dos utilizadores separa resíduos diariamente; faixa etária predominante 18--25; plástico e embalagens de metal (ecoponto amarelo) são os mais reciclados.
    \item \textbf{Experiência com aplicações:} Utilizadores confortáveis com aplicações novas e uso regular no smartphone.
    \item \textbf{Usabilidade pós-teste:} Melhoria na utilidade percebida e na navegação; identificação de \textbf{inconsistências} pontuais a corrigir.
\end{itemize}

\subsection{Notas finais}
\begin{itemize}
    \item O foco principal do projeto esteve nos \textbf{estudos de campo} e na \textbf{validação empírica}; as alterações na aplicação foram \textbf{pontuais} e orientadas à qualidade dos dados e ao alinhamento conceptual.
    \item Próximos passos: correção das inconsistências reportadas, simplificação adicional de fluxos e expansão dos instrumentos de avaliação.
\end{itemize}