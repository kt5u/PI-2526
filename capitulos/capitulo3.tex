%==========================================================================
% @chapter: Estudos de Campo
% @description: Descrição da fase experimental, metodologias de recolha 
%              de dados e análise quantitativa/qualitativa do feedback.
%==========================================================================
\chapter{Estudos de Campo}
\label{cap:estudos_campo}

%==========================================================================
% @section: Metodologia e Dinâmica do Estudo
% @description: Descrição detalhada da abordagem metodológica e dinâmica do estudo.
%==========================================================================
\section{Metodologia e Dinâmica do Estudo}

Este capítulo descreve a componente prática e experimental do projeto, focada na interação direta com os utilizadores finais da aplicação. O principal objetivo do estudo de campo foi validar a utilidade da solução desenvolvida, medir a evolução da experiência do utilizador e avaliar o impacto real das melhorias implementadas na versão 2024/2025 em comparação com as iterações anteriores.

Para garantir a fiabilidade dos resultados e compreender a evolução da perceção dos utilizadores, a metodologia de investigação assentou na realização de dois momentos distintos de auscultação:

\begin{itemize}
    \item \textbf{Questionário Pré-teste:} Aplicado antes da interação prolongada com a versão atual, com o intuito de aferir as expectativas, hábitos de reciclagem e identificar problemas recorrentes baseados em experiências anteriores com o sistema;
    \item \textbf{Questionário Pós-teste:} Realizado após a fase de experimentação ativa, focado na avaliação da usabilidade, na satisfação geral e na eficácia das novas funcionalidades introduzidas nesta última versão.
\end{itemize}

Para o tratamento e análise dos resultados, recorreu-se ao \textbf{PowerBI}, onde foi concebida uma \textit{dashboard} interativa. Esta solução permitiu otimizar a visualização dos dados e fundamentar a análise crítica das métricas de usabilidade e satisfação.

%==========================================================================
% @section: Questionário de Avaliação da Aplicação ReciclaApp
% @description: Detalhe do instrumento de colheita de dados, objetivos 
%              científicos e enquadramento no modelo UTAUT2.
%==========================================================================
%==========================================================================
% @section: Questionário de Avaliação Final (Pós-teste) da ReciclaApp
% @description: Detalhe do instrumento de colheita de dados final, 
%              focado na experiência de uso e aceitação tecnológica.
%==========================================================================
%==========================================================================
% @section: Questionário de Diagnóstico e Satisfação Inicial (Pré-teste)
% @description: Detalhe do instrumento de auscultação inicial aplicado para 
%              aferir hábitos de reciclagem e expectativas dos utilizadores.
%==========================================================================
\section{Questionário Pré-teste: Diagnóstico e Expectativas do Utilizador}
\label{sec:questionario_pre_teste}

Este capítulo descreve o instrumento de inquérito aplicado na fase inicial do ensaio de campo da \textit{ReciclaApp}. Sendo um \textbf{Questionário Pré-teste}, o seu propósito é estabelecer uma linha de base (\textit{baseline}) sobre os comportamentos de reciclagem da comunidade académica e as suas expectativas em relação à solução tecnológica proposta.

%==========================================================================
% @subsection: Objetivos do Inquérito
% @description: Definição das metas de avaliação inicial para professores, 
%              estudantes e funcionários do IPBeja.
%==========================================================================
\subsection{Objetivos}
O principal objetivo deste inquérito é avaliar o perfil e o nível de prontidão dos utilizadores da \textit{ReciclaApp} — abrangendo docentes, estudantes e funcionários do IPBeja. Através deste diagnóstico, pretende-se compreender até que ponto a aplicação antecipa as necessidades dos utilizadores e identificar, precocemente, aspetos críticos que requerem atenção no desenvolvimento.

%==========================================================================
% @subsection: Metodologia de Medição
% @description: Detalhe das métricas utilizadas, incluindo a Escala Likert 
%              e o System Usability Scale (SUS).
%==========================================================================
\subsection{Instrumentos de Medição e Escalas}

Para garantir a validade estatística e a comparabilidade dos dados, foram adotados dois instrumentos principais de medição:

\subsubsection{Escala de Resposta Likert}
A maior parte das questões de opinião utiliza uma escala \textbf{Likert de 5 pontos}, permitindo quantificar o grau de concordância dos participantes:


\begin{description}
  \item[1] Discordo totalmente
  \item[2] Discordo
  \item[3] Nem concordo nem discordo
  \item[4] Concordo
  \item[5] Concordo totalmente
\end{description}

\subsubsection{System Usability Scale (SUS)}
Como complemento, foi aplicado o \textit{System Usability Scale} (SUS) \cite{sus}. Este instrumento é composto por 10 afirmações respondidas numa escala de 1 a 5. A pontuação final do SUS é posteriormente convertida para uma escala de 0 a 100, seguindo a metodologia padrão da indústria para avaliar a perceção de usabilidade global de um sistema.

%==========================================================================
% @subsection: Acesso ao Formulário
% @description: Referência para a versão digital do questionário de 
%              diagnóstico inicial.
%==========================================================================
\subsection{Acesso ao Questionário}
O questionário de diagnóstico foi disponibilizado através da plataforma \textit{Google Forms}, podendo ser consultado na integra através da seguinte ligação: 

\begin{quote}
    \textbf{URL (Pré-teste):} \url{https://forms.gle/2bKXGUvXgRwZpfGE9}
\end{quote}

\section{Questionário Pós-teste: Avaliação de Experiência e Aceitação}
\label{sec:questionario_pos_teste}

Este capítulo detalha o instrumento de colheita de dados final, designado por \textbf{Questionário Pós-teste}, aplicado após o período de experimentação da \textit{ReciclaApp}. Ao contrário do diagnóstico inicial, este questionário foca-se na experiência real e vivida pelos utilizadores com a versão 2024/2025.

%==========================================================================
% @subsection: Objetivos do Pós-teste
% @description: Medição da utilidade percebida e intenção de uso futuro 
%              após a fase de testes.
%==========================================================================
\subsection{Objetivos e Validação de Uso}
O principal objetivo deste questionário é compreender como os docentes e funcionários do IPBeja avaliaram a aplicação após a sua utilização efetiva. Interessa-nos validar se a \textit{ReciclaApp} demonstrou utilidade prática no quotidiano, se a navegação foi considerada intuitiva e se o sistema de registo manual de objetos se revelou mais fiável do que as soluções automáticas anteriores.

%==========================================================================
% @subsection: Enquadramento no Modelo UTAUT2
% @description: Aplicação das dimensões de aceitação tecnológica para 
%              validar a robustez da aplicação.
%==========================================================================
\subsection{Enquadramento Teórico (UTAUT2)}
Para garantir o rigor científico, o questionário foi estruturado em dimensões que medem a aceitação tecnológica real, com base no modelo \textbf{UTAUT2} \cite{fithriya2019user}. 

Nesta fase de pós-teste, as métricas focam-se em:
\begin{itemize}
    \item \textbf{Hábito:} A facilidade com que a reciclagem via app se tornou uma rotina;
    \item \textbf{Valor de Preço/Esforço:} Se os benefícios de utilizar a app superam o esforço de registo;
    \item \textbf{Intenção Comportamental:} A probabilidade real de o utilizador continuar a usar a \textit{EcoSepara} no futuro.
\end{itemize}

%==========================================================================
% @subsection: Acesso ao Questionário Final
% @description: Link para a versão do Google Forms utilizada para a 
%              recolha final de feedback.
%==========================================================================
\subsection{Acesso ao Questionário Pós-teste}
Para garantir a integridade dos dados e facilitar a sua posterior análise na \textit{dashboard} de \textbf{PowerBI}, o questionário foi disponibilizado digitalmente:

\begin{quote}
    \textbf{URL (Pós-teste):} \url{https://forms.gle/SC8j9m1sgmhuotR8A}
\end{quote}

%==========================================================================
% @section: Interpretação dos Resultados dos Questionários
% @description: Análise crítica dos dados recolhidos, detalhando o fluxo 
%              de processamento desde a recolha até à visualização.
%==========================================================================
\section{Interpretação dos Resultados dos Questionários}
\label{sec:interpretacao_resultados}

Após a fase de recolha e submissão dos questionários, procedeu-se a uma análise detalhada das respostas obtidas. Este processo foi fundamental para converter dados brutos em informação relevante para a avaliação do projeto. Para assegurar o rigor analítico, recorreu-se à metodologia \textbf{ETL} (\textit{Extract, Transform, Load}), permitindo que os dados provenientes do \textit{Google Forms} fossem processados de forma sistemática. 

Os dados foram inicialmente exportados em formato \textit{Comma-Separated Values} (CSV) e, posteriormente, sujeitos a procedimentos de tratamento e modelação, garantindo a sua integridade, consistência e adequação para a fase de interpretação estatística.

%==========================================================================
% @subsection: Processo ETL (Extract, Transform, Load)
% @description: Descrição das etapas de tratamento de dados para garantir 
%              a qualidade da análise estatística.
%==========================================================================
\subsection{Processo ETL}
O fluxo de processamento de dados foi segmentado em três etapas principais, garantindo que a análise final refletisse com precisão o \textit{feedback} dos utilizadores:



\begin{itemize}
  \item \textbf{Extract (Extração):} Nesta fase inicial, os dados brutos foram extraídos da plataforma \textit{Google Forms} e exportados para um ficheiro de suporte (CSV), servindo como repositório primário para o processamento subsequente.
  
  \item \textbf{Transform (Transformação):} Considerada a etapa crítica para a qualidade da informação, os dados foram sujeitos a uma limpeza profunda. Este procedimento incluiu a remoção de registos duplicados, a retificação de erros de preenchimento e a normalização de formatos, assegurando que as variáveis estivessem prontas para a aplicação de modelos analíticos.
  
  \item \textbf{Load (Carregamento):} Os dados devidamente tratados foram carregados no ambiente de análise (Power BI). Nesta etapa, realizaram-se os testes estatísticos e a geração de métricas descritivas que fundamentam a interpretação dos resultados e as conclusões do estudo.
\end{itemize}

\subsection{Resultados do questionário pré-teste: Separação de Resíduos}

Os resultados do questionário pré-teste foram fundamentais para compreender o contexto inicial dos utilizadores e as suas expectativas em relação à aplicação \textit{EcoSepara}. A análise das respostas permitiu identificar áreas de melhoria e ajustar a abordagem do projeto antes da implementação.

Os principais resultados incluem:

\begin{itemize}
    \item \textbf{Perfil dos Utilizadores:} A maioria dos participantes identificou-se como utilizadores frequentes de aplicações nos seus dispositivos móveis e demonstraram uma predisposição para a utilização de soluções digitais.
    \item \textbf{Expectativas:} Os utilizadores esperam que a aplicação facilite o processo de reciclagem, tornando-o mais intuitivo e acessível.
    \item \textbf{Preocupações:} Foram levantadas preocupações quanto à complexidade do registo e à necessidade de suporte contínuo durante a utilização da aplicação.
\end{itemize}

\paragraph{Principais destaques da primeira dashboard}
Observa-se que a maioria dos inquiridos refere separar resíduos diariamente. A faixa etária predominante situa-se entre os \textbf{18 e 25 anos}, evidenciando maior participação dos utilizadores mais jovens. Quanto aos materiais, o \textbf{plástico e embalagens de metal} (ecoponto amarelo) surge como o fluxo de resíduos \textbf{mais reciclado} pelos participantes.

\begin{figure}[htbp]
    \centering
    \includegraphics[width=\linewidth]{figuras/dashboard.png}
    \caption{Dashboard do Power BI utilizada para a análise dos resultados do pré-teste.}
    \label{fig:dashboard_preteste}
\end{figure}

\subsection{Resultados do questionário pré-teste: Experiência com Aplicações}

Esta segunda \textit{dashboard} foca-se na experiência dos utilizadores com aplicações móveis, incluindo familiaridade com apps de sustentabilidade, frequência de uso, perceção de facilidade de utilização e satisfação geral.

\paragraph{Principais destaques da segunda dashboard}
Constata-se que a maior parte dos utilizadores sente-se \textbf{confortável} ao usar aplicações novas e \textbf{costuma utilizar apps no smartphone} com regularidade, indicando elevada literacia digital e predisposição para adoção de novas soluções.

Os principais indicadores visualizados incluem:
\begin{itemize}
    \item \textbf{Familiaridade com Apps:} Nível de contacto prévio com aplicações semelhantes.
    \item \textbf{Frequência de Uso:} Periodicidade de uso no quotidiano académico/profissional.
    \item \textbf{Facilidade de Utilização:} Avaliação de navegação, clareza de interface e esforço cognitivo.
    \item \textbf{Satisfação Geral:} Sentimento global face à experiência com a versão atual.
\end{itemize}
\begin{figure}[htbp]
    \centering
    \includegraphics[width=\linewidth]{figuras/dashboard2.png}
    \caption{Dashboard do Power BI focada na experiência dos utilizadores com aplicações.}
    \label{fig:dashboard_apps}
\end{figure}

\subsection{Interpretação dos Resultados do Questionário Pós-teste}

A análise do pós-teste evidencia melhorias na perceção de utilidade e na usabilidade da \textit{ReciclaApp} após a fase de experimentação. Em geral, os participantes reportaram que:
\begin{itemize}
    \item \textbf{Utilidade Percebida:} A aplicação facilitou o registo e a separação de resíduos no quotidiano, com ganhos de eficiência face a versões anteriores.
    \item \textbf{Usabilidade:} A navegação foi considerada mais intuitiva, com menor esforço cognitivo e menor incidência de erros durante o registo.
    \item \textbf{Confiabilidade do Registo Manual:} O método manual mostrou-se mais fiável e transparente do que as soluções automáticas previamente testadas.
    \item \textbf{Hábito e Continuidade:} Observou-se tendência de incorporação da app na rotina, com \textbf{alta intenção de uso futuro}.
\end{itemize}

\paragraph{Inconsistências observadas}
Apesar dos ganhos reportados, alguns utilizadores assinalaram \textbf{inconsistências} pontuais na aplicação (p. ex., comportamento inesperado em certos fluxos de registo e discrepâncias de feedback), sugerindo necessidade de correções incrementais.

\begin{figure}[htbp]
    \centering
    \includegraphics[width=0.9\linewidth]{figuras/app_inconsistencias.png}
    \caption{Registo de inconsistências observadas pelos utilizadores no pós-teste.}
    \label{fig:ainconsistencias_app}
\end{figure}

\begin{figure}[htbp]
    \centering
    \includegraphics[width=0.9\linewidth]{figuras/facilidade_app.png}
    \caption{Indicadores de facilidade de utilização das aplicações pelos utilizadores.}
    \label{fig:facilidade_app}
\end{figure}

No conjunto, os resultados do pós-teste validam a evolução da versão 2024/2025, sustentando a sua adoção contínua e apontando oportunidades incrementais de melhoria (p. ex., simplificação adicional de fluxos de registo e feedback contextual).

