\chapter{Refinamentos na Aplicação}

Após a realização do estudo de campo com utilizadores reais, foi possível recolher feedback direto sobre a utilização
 da aplicação em contexto prático. A análise das interações, dificuldades identificadas e sugestões apresentadas pelos 
 utilizadores permitiu identificar oportunidades claras de melhoria. Com base nesse feedback, 
 foram implementados diversos refinamentos com o objetivo de tornar a aplicação
  mais próxima do objetivo que ela representa, eficaz e alinhada com as necessidades reais dos seus utilizadores.

\clearpage
\section{Alteração de nome e conceito da aplicação}
\subsection{Objetivo}
Alterar o nome e a conceito da aplicação para refletir de forma mais adequada a sua finalidade, passando do conceito de 
reciclagem para o de separação de resíduos.

\subsection{Alterações Realizadas}
O nome da aplicação foi alterado de \textit{ReciclaApp} para \textit{EcoSepara}, bem como toda a terminologia associada, 
substituindo os termos e expressões onde se usava o conceito de \textit{reciclagem} pelo conceito de \textit{separação de resíduos}, 
transitando de uma abordagem generalista à reciclagem para uma abordagem específica e orientada à separação de resíduos.


\subsection{Benefícios}
Estas alterações resultam num nome mais intuitivo e alinhado com os objetivos da aplicação, reforçando o foco 
na ação direta do utilizador na separação de residuos e não no processo de reciclagem.

\clearpage
\section{Sistema de Monitorização dos Pedidos do Utilizador}
\subsection{Objetivo}
Permitir a identificação do utilizador que submeteu cada pedido de adição de novo código de barras para introdução na base de dados, 
de modo a haver rastreabilidade dos pedidos dos utilizadores.

\subsection{Implementação}
Foram adicionados campos de identificação do utilizador ao modelo de dados, capturada automaticamente a informação do 
utilizador autenticado no momento da submissão do pedido de adição de código de barras e apresentada essa informação,
 em cada código de barras a aceitar na interface do administrador, como se vê na figura seguinte.

\begin{figure}[H]
    \centering
    \includegraphics[width=0.4\textwidth]{figuras/userPedido.png}
    \caption{Aceitação de código de barras - Administrador}
    \label{fig:userPedido}
\end{figure}


\subsection{Benefícios}
Este sistema garante rastreabilidade completa dos pedidos, sabendo quem fez o pedido e mantém compatibilidade 
retroativa com os dados já existentes na base de dados.

\clearpage
\section{Sistema de Localização}
\subsection{Objetivo}
Registar o local onde a separação foi realizada, distinguindo entre ações efetuadas dentro e fora do IPBEJA.

\subsection{Implementação}
Foi introduzido um mecanismo de seleção obrigatória da localização no momento do registo da separação, com validação e 
armazenamento da informação na base de dados.


\begin{figure}[H]
    \centering
    \includegraphics[width=0.25\textwidth]{figuras/Localização.png}
    \caption{Escolha da Localização da Separação}
    \label{fig:localizacao}
\end{figure}


\subsection{Benefícios}
A distinção clara entre a escola do utilizador e o local da separação permite análises geográficas mais precisas e evita
 o registo de dados incompletos.

\section{Gráfico de Comparação}
\subsection{Objetivo}
Visualizar graficamente a distribuição das separações realizadas dentro e fora do IPBeja, na página de estatísticas do administrador.

\subsection{Implementação}
Foi desenvolvido um gráfico comparativo com barras de progresso com indicação 
de percentagens e valores absolutos, como se pode observar na seguinte figura.

\begin{figure}[H]
    \centering
    \includegraphics[width=0.25\textwidth]{figuras/EstatisticasAdmin.png}
    \caption{Página de Estatísticas do Administrador}
    \label{fig:interfaceEstatisticasAdmin}
\end{figure}

\subsection{Benefícios}
A visualização imediata facilita a interpretação dos dados globais e mantém compatibilidade com registos antigos.

\subsection{Separação Conceptual: Escola vs Localização}
\subsubsection{Conceitos Distintos}
O campo \textit{escola} identifica a escola de origem do utilizador e é fixo no perfil, enquanto o campo \textit{localização}
 indica onde a separação foi realizada e varia em cada registo.

\subsubsection{Benefícios}
Esta separação conceptual aumenta a flexibilidade analítica e permite identificar padrões de comportamento por escola e por local.

\clearpage
\section{Conclusões Finais dos Refinamentos}

Todas as funcionalidades foram implementadas e devidamente testadas. O rebranding da aplicação, o sistema 
de tracking de utilizadores, o registo de localização, os filtros estatísticos e os 
gráficos comparativos encontram-se totalmente funcionais na nova versão da aplicação.

Os refinamentos introduzidos melhoram significativamente a qualidade dos dados, a experiência do utilizador e a capacidade de
 análise, mantendo total compatibilidade retroativa e estabilidade da aplicação.
