\chapter{Ano Letivo 2024/2025}
\label{cap4}
\section{Análise do projeto anterior(23/24)}
Os alunos iniciaram a análise do relatório e da aplicação desenvolvidada no ano letivo anterior (23/24). Analisaram os casos de Uso implementados, a base de dados e tanto as funcionalidades como as interfaces da aplicação.
Retiraram algumas conclusões sobre o trabalho realizado principalmente no que diz respeito ás interfaces que consideraram que poderiam ser melhoradas em termos de usabilidade e experiência do utilizador.
\section{Sugestões de melhoria e implementação}
Os alunos sugeriram algumas melhorias e procederam para a sua implementação.
Como a otimização da base de dados FireStore, inserção de scripts para a filtragem e obtenção de imagens dos produtos premitindo assim uma expansao significastiva da base de dados.
Alterações na interface da página principal e no sitema de navegação da aplicação tornarndo---a mais intuitiva e simples de utilizar, melhorias no calculo das emissoes e na energia poupada com a reciclagem.  
\subsection{Nova implemntação na base de dados FireStore}
Na versão anterior da base de dados (23/24), as imagens dos códigos de barras eram armazenadas em cada registo de reciclagem, o que fazia com que cada registo tivesse uma foto diferente do produto reciclado. Para resolver este problema, foi criada uma nova coleção na base de dados, associando a imagem ao código de barras do produto e não ao registo de reciclagem, garantindo assim que cada código de barras possui apenas uma imagem associada.

Além disso, foi desenvolvido um script em Python que filtra os produtos da base de dados OpenFoodFacts e obtém as imagens dos produtos através da API disponibilizada pela plataforma, com foco nos produtos portugueses.

\subsection{Alterações no valores de emissões}
Os alunos encontraram na análise que os valores de emissões associados a cada tipo de material reciclado não estavam explicitados no relatório anterior (23/24). Decidiram, portanto, atualizá-los com base em fontes mais recentes, recorrendo a sites de organizações confiáveis. Para maior precisão, começaram a utilizar valores médios para os materiais reciclados, reconhecendo que, por exemplo, uma garrafa de 1,5 litros pode não ter sempre o mesmo peso.
\subsection{Melhorias na Interface e Navegação}
Foram realizadas melhorias na interface em ajuste a experiência do utilizador, nomeadamente no registo da reciclagem, em que os alunos optaram por retirar o reconhecimento automático dos objetos, por considerarem pouco fiável. Assim o utilizador tira a foto ao produto e insere manualmente o código de barras, fazendo o registo todo manualmente como podemos ver na figura a baixo.

\begin{figure}[H]
    \centering
    \includegraphics[width=0.3\textwidth]{figuras/registo.png}
    \caption{Interface de registo de reciclagem}
    \label{fig:interfaceRegisto}
\end{figure}

Foram efetuadas melhorias na página principal da aplicação, tanto na área do administrador como na do reciclador. O ecrã dedicado apenas à navegação foi removido; agora, após o login, o reciclador é direcionado para uma página que apresenta a lista das suas reciclagens e os totais de emissões poupadas. Por sua vez, o administrador é encaminhado diretamente para a página de gestão dos pedidos de inserção de códigos de barras sugeridos pelos recicladores.

\begin{figure}[H]
    \centering
    \includegraphics[width=0.6\textwidth]{figuras/inicial2425.png}
    \caption{Interface da página principal do reciclador}
    \label{fig:interfacePrincipal}
\end{figure}

Foram também efetuadas melhorias na navegação da aplicação, incluindo a implementação de um menu inferior com as principais funcionalidades, tornando o percurso do utilizador mais intuitivo e simples. Além disso, foram realizadas alterações no tema da aplicação, mantendo apenas o tema claro (lightTheme), e passou a ser possível visualizar a password no ecrã de login.


\section{Conclusão e trabalhos futuros}
Os alunos concluíram que ainda há melhorias a realizar para tornar a aplicação mais completa e otimizada. Realizaram melhorias na base de dados Firestore, procurando otimizar ao máximo a sua estrutura, além de aprimoramentos na interface e na navegação, tornando a aplicação mais intuitiva e fácil de usar. Também atualizaram os valores das emissões poupadas com a reciclagem.

Para trabalhos futuros, sugerem a implementação de métricas adicionais na base de dados, como litros, mililitros e quilogramas reciclados, a criação de um sistema de recompensas para incentivar a participação dos utilizadores, a unificação das páginas de registo, a integração de ferramentas que possibilitem o reconhecimento automático de objetos com maior precisão e a melhoria geral das interfaces.