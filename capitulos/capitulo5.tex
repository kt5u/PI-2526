\chapter{Testes e Avaliação Funcional da Aplicação}
\label{cap5}

\section{Avaliação Funcional das Versões da Aplicação}
Foram avaliadas as três versões da aplicação (2022/2023, 2023/2024 e 2024/2025). 
Na comparação inicial entre as versões de 2022/2023 e 2023/2024, verificou-se que a versão de 2023/2024 
apresenta uma interface mais eficiente nas funcionalidades de administrador, 
bem como pequenas melhorias gerais ao nível da interface. 
Estas conclusões resultam tanto dos testes realizados como da análise apresentada nos capítulos anteriores.

Posteriormente, procedeu-se à comparação entre as versões de 2023/2024 (a melhor versão anterior a 2024/2025) 
e 2024/2025, constatando-se uma evolução significativa nesta última. A versão 2024/2025 mostrou-se 
notavelmente mais robusta, destacando-se como a mais completa e estável entre todas as versões avaliadas.
Nesta versão, é melhorada significativamente a interface gráfica e a forma como navegamos na aplicação,   
é melhorada a forma como se introduzem os objetos na base de dados e verificaram-se menos falhas
ao utiliza-la.

\clearpage
\section{Análise Comparativa: Versões 2022/2023 e 2023/2024}

Não se verificaram alterações significativas ao nível da interface gráfica entre as duas versões, 
sendo as modificações mais relevantes focadas no backend, conforme descrito nos capítulos anteriores. 
Na versão de 2023/2024, destaca-se a introdução de duas novas opções na página inicial do administrador, 
que não estavam presentes na versão anterior.

\subsection{Página Inicial do Administrador}

A próxima figura mostra a página inicial do administrador da versão 2022/2023:

\begin{figure}[H]
    \centering
    \includegraphics[width=0.3\textwidth]{figuras/pagina-inicial-admin2223.jpeg}
    \caption{Página inicial do administrador - versão 2022/2023}
    \label{fig:admin-2223}
\end{figure}

\clearpage
A próxima figura mostra a página inicial do administrador da versão 2023/2024 onde
podemos observar as duas opções introduzidas nesta versão. "Validar Código", através 
da qual é apresentada uma lista de todos os códigos de barras submetidos pelos 
recicladores para aprovação e posterior inserção na base de dados e "Adicionar Código" que 
permite ao administrador adicionar novos códigos de barras à base de dados, à semelhança 
do reciclador.

\begin{figure}[H]
    \centering
    \includegraphics[width=0.4\textwidth]{figuras/pagina-inicial-admin2324.jpeg}
    \caption{Página inicial do administrador - versão 2023/2024}
    \label{fig:admin-2324}
\end{figure}

\clearpage
\section{Análise Comparativa: Versões 2023/2024 e 2024/2025}

Na versão de 2024/2025, observam-se melhorias significativas ao nível da interface 
gráfica e da navegação, quando comparada com a versão anterior (2023/2024). 
Para além da nova base de dados, que agora permite não só receber a fotografia 
do código de barras mas também a imagem do objeto a reciclar, a interface inclui 
um novo menu de navegação (conforme referido nos capítulos anteriores) e diversas 
melhorias gráficas e funcionais.

\subsection{Página Inicial e Menu de Navegação}

A página inicial, que anteriormente tinha os botões das funções a desempenhar por cada utilizador
foi substituida, no caso do reciclador, pela página onde consegue ver as suas reciclagens 
e o seu impacto na reciclagem, e no caso do administrador, a página inicial da versão anterior, 
foi substituída pela página de aprovação de produtos e os seus respetivos códigos para inserção 
na base de dados. A navegação entre as diversas opções passou a ser feita através do menu de navegação 
localizado na parte de baixo do ecrâ da aplicação.

\clearpage
Como exemplo, as próximas figuras mostram as diferenças da página inicial do reciclador nas 
duas versões analisadas.

\begin{figure}[H]
    \centering
    \includegraphics[width=0.4\textwidth]{figuras/pagina-inicial-reciclador2324.jpeg}
    \caption{Página inicial do reciclador - versão 2023/2024}
    \label{fig:reciclador-2324}
\end{figure}

\begin{figure}[H]
    \centering
    \includegraphics[width=0.4\textwidth]{figuras/pagina-inicial-reciclador2425.jpeg}
    \caption{Página inicial do reciclador - versão 2024/2025}
    \label{fig:reciclador-2425}
\end{figure}

\clearpage
\subsection{Introdução de Reciclagem pelo Reciclador}

Na secção de introdução da reciclagem, foi removida a opção de reconhecimento de objetos, 
uma vez que não funcionava conforme esperado nas versões anteriores. Sempre que era selecionada, 
a aplicação apresentava um erro e encerrava inesperadamente.

As próximas figuras mostram as diferenças da página inicial da reciclagem nas duas versões analisadas.

\begin{figure}[H]
    \centering
    \includegraphics[width=0.4\textwidth]{figuras/introduzir-reciclagem2324.jpeg}
    \caption{Introduzir Reciclagem - versão 2023/2024}
    \label{fig:reciclagem-2324}
\end{figure}

\clearpage
Na seguinte figura observa-se que o botão de reconhecimento de objeto foi removido da interface.

\begin{figure}[H]
    \centering
    \includegraphics[width=0.4\textwidth]{figuras/introduzir-reciclagem2425.jpeg}
    \caption{Introduzir Reciclagem - versão 2024/2025}
    \label{fig:reciclagem-2425}
\end{figure}

\clearpage
\subsection{Sugestão de Código pelo Reciclador}

Na introdução de uma reciclagem por parte do reciclador, caso o código de barras selecionado 
não exista na base de dados, o utilizador é convidado a sugerir a sua adição, introduzindo os
 restantes detalhes do produto. 
Na versão de 2023/2024, não existe opção de fotografar o produto, sendo apenas possível guardar a fotografia 
do código de barras.

Na versão de 2024/2025, esta funcionalidade foi melhorada com a introdução da opção de anexar uma fotografia 
do produto na íntegra, o que resultou numa melhoria significativa tanto na apresentação das reciclagens 
já realizadas ao reciclador como no processo de aprovação de códigos de barras por parte do administrador.

Na próxima figura é possível verificar que não existe opção de tirar fotografia ao produto na versão 
de 2023/2024 ao sugerir adição de código.

\begin{figure}[H]
    \centering
    \includegraphics[width=0.3\textwidth]{figuras/sugerir-codigo2324.jpeg}
    \caption{Sugerir Código - versão 2023/2024}
    \label{fig:sugerir-2324}
\end{figure}

\clearpage
Na seguinte figura observa-se que é possível tirar uma fotografia ao produto além de guardar o código de barras 
e de diversas melhorias na interface.

\begin{figure}[H]
    \centering
    \includegraphics[width=0.4\textwidth]{figuras/sugerir-codigo2425.jpeg}
    \caption{Sugerir Código - versão 2024/2025}
    \label{fig:sugerir-2425}
\end{figure}

\clearpage
Na próxima figura é apresentada a página de reciclagens realizadas por parte do reciclador onde se consegue 
observar a foto do código de barras em cada reciclagem na versão 2023/2024.

\begin{figure}[H]
    \centering
    \includegraphics[width=0.4\textwidth]{figuras/reciclagens-feitas2324.jpeg}
    \caption{Reciclagens Feitas - versão 2023/2024}
    \label{fig:reciclagens-2324}
\end{figure}

\clearpage
Na figura seguinte, observa-se que, nos itens da lista, é exibida a fotografia do produto.

\begin{figure}[H]
    \centering
    \includegraphics[width=0.4\textwidth]{figuras/reciclagens-feitas2425.jpeg}
    \caption{Reciclagens Feitas - versão 2024/2025}
    \label{fig:reciclagens-2425}
\end{figure}

\clearpage
\subsection{Aceitação de Códigos de Barras pelo Administrador}

Na versão de 2023/2024, na aceitação dos códigos de barras por parte do administrador, apresenta 
uma lista com apenas os códigos de barras e ao clicar em cada item da lista, apresenta os detalhes 
do produto, sem foto, e apenas com a presença de um botão de validar e nenhum botão de rejeitar, 
ficando com uma lista pendente sem poder eliminar os items dessa lista.

Na versão de 2024/2025, a página de aceitação de códigos de barras a inserir na base de dados, 
cada item tem a foto do produto, apresenta as unidades das métricas do produto e além de podermos
 aceitar o código de barras, temos a opção de rejeita-lo, eliminando o produto da lista.

Na próxima figura é possível verificar a lista códigos de barras pendentes de aceitação e os detalhes de um dos
produtos.

\begin{figure}[H]
    \centering
    \includegraphics[width=0.3\textwidth]{figuras/aceitar-codigo2324.jpeg}
    \caption{Aceitar Código de Barras - versão 2023/2024}
    \label{fig:aceitacao-2324}
\end{figure}

\clearpage
Na seguinte figura observa-se a presença da foto em cada item da lista e uma interface gráfica 
completamente renovada, com um botão de rejeitar o produto.

\begin{figure}[H]
    \centering
    \includegraphics[width=0.4\textwidth]{figuras/aceitar-codigo2425.jpeg}
    \caption{Aceitar Código de Barras - versão 2024/2025}
    \label{fig:aceitacao-2425}
\end{figure}

\clearpage
\section{Conclusão Final da Análise das Versões}

A análise comparativa das versões 2022/2023, 2023/2024 e 2024/2025 evidencia uma evolução contínua 
e significativa da aplicação ao longo do tempo. A versão 2022/2023 apresentou uma boa base funcional, 
posteriormente aprimorada na versão 2023/2024, que integrou melhorias relevantes na 
interface de administrador, pequenas correções na interface e melhorias ao nivel do backend. 
No entanto, foi na versão 2024/2025 que se observaram as mudanças mais substanciais, tanto ao nível 
da interface gráfica como na robustez das funcionalidades, na estrutura da base de dados e na estabilidade 
geral da aplicação.

As melhorias introduzidas, incluindo o novo sistema de navegação, a possibilidade de anexar 
fotografias completas dos produtos, a otimização dos processos de validação e a redução das falhas 
operacionais, consolidam esta versão como a mais completa, estável e funcional entre todas as 
analisadas.

Assim, conclui-se que a versão 2024/2025 é a mais adequada para avançar para fases mais 
aprofundadas de testes, garantindo melhores condições de avaliação e continuidade de desenvolvimento.














