\chapter{Questionário de Avaliação da Aplicação ReciclaApp}
\label{cap7}

Este capítulo descreve o questionário aplicado durante o ensaio de campo da ReciclaApp.
O principal objetivo deste questionário é perceber como é que professores, docentes e funcionários do IPBeja realmente utilizam a aplicação e o que pensam sobre ela. Interessa-nos saber se a ReciclaApp lhes faz sentido no dia a dia, se é fácil de usar, se ajuda de alguma forma no processo de reciclagem e se existe vontade de continuar a utilizá-la no futuro. Para organizar estas opiniões, recorremos a várias dimensões que são normalmente usadas em estudos sobre aceitação de tecnologia, como as que fazem parte do modelo UTAUT2 (Unified Theory of Acceptance and Use of Technology 2). Isto permite reunir informação de forma mais consistente e comparar melhor a experiência dos diferentes utilizadores.
As respostas recolhidas são muito importantes para entendermos o que está bem encaminhado e o que ainda precisa de ser melhorado. Estas opiniões ajudam não só a confirmar se as funcionalidades atuais estão a cumprir o que se pretende, mas também a orientar o trabalho futuro. Assim, a aplicação pode ir evoluindo com base em feedback real de quem a utiliza.
Na secção seguinte encontra-se o questionário modelo completo usado no estudo, também disponível em anexo como documento separado. 
Para facilitar o questionário, foi realizado um questionário idêntico no GoogleForms que poderá ser visualizado no seguinte link:
URL.- \url{https://forms.gle/SC8j9m1sgmhuotR8A}